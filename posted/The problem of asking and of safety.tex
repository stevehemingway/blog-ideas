\documentclass{article}
\begin{document}

\section{Rethinking the Precautionary Principle: Accepting Avoidable Deaths as a Reality}

The precautionary principle, which dictates that we should avoid any action that might potentially cause harm, has been a cornerstone of decision-making in many areas, including politics, law, and science. However, this principle is based on the dubious assumption that avoiding death and harm should be trump all other considerations. In reality, individuals do not always prioritize avoiding death above all else, and governments should not do so either. If I have an extra beer in the evening, I probably am shortening, in quality-adjusted time, my life by a few hours or a few days. Cheers!

\subsection{The Irrationality of Zero-Risk Mentality}

Consider the example of road safety. While it's true that wearing a seatbelt reduces the risk of death in a car accident, it's not the only factor that determines our overall safety. In fact, the cost of implementing and maintaining safety measures, such as road infrastructure and vehicle safety features, is substantial. If we were to allocate an infinite budget to road safety, we could potentially eliminate all road deaths. However, this would come at the cost of diverting resources away from other important areas, such as healthcare, education, and economic development.

\subsection{The Trade-Off Between Safety and Other Goods}

Individuals make trade-offs between safety and other goods and services all the time. For example, people may choose to drive to work instead of taking public transportation, despite the higher risk of accident, because it's more convenient. Similarly, people may choose to engage in risky activities, such as skydiving or rock climbing, or unsafe sex, or taking illicit drugs, because they derive utility from the thrill and adventure. 

Governments should also make similar trade-offs when allocating resources. For instance, investing in healthcare may save more lives than investing in road safety. By prioritizing one area over another, governments are implicitly making a judgment about the relative value of different lives and the utility derived from different activities. Policy can be regressive, and different income groups have different discount rates. Not all of us want to live to a hundred, but be impoverished by the medical bills.

\subsection{The Statistical Case for Accepting Avoidable Deaths}

From a statistical perspective, it's clear that eliminating all risk is not only impossible but also undesirable. The cost of achieving zero risk would be prohibitively expensive, and the opportunity cost of diverting resources away from other areas would be substantial.

For example, in the UK, the annual cost of road accidents is estimated to be around £15 billion. While reducing this number is desirable, it's not the only consideration. If we were to allocate an additional £10 billion to road safety measures, we might be able to reduce the number of road deaths by 10\%. However, this would mean diverting resources away from other areas, such as healthcare, where the same £10 billion could potentially save more lives. In Wales, the national speed limit has been reduced to 20 mph. I'm sure there are fewer accidents, but equally there are slower journeys. 

\section{Maximizing Utility in a Mixed Economy}

In a mixed economy, where many goods and services are provided by the state, maximizing utility requires careful consideration of the trade-offs between different areas. By acknowledging that avoiding death is not the only consideration, governments can make more informed decisions about how to allocate resources. 

For instance, in the case of healthcare, governments might prioritize treatments that offer the greatest quality-adjusted life years (QALYs) per dollar spent. Similarly, in the case of transportation, governments might prioritize investments that offer the greatest reduction in travel time or improvement in air quality per dollar spent.

\section{Conclusion}

The precautionary principle is based on a flawed assumption that avoiding death and harm is the highest utility consumable. In reality, individuals and governments must make trade-offs between safety and other goods and services. Voters have been conditioned to vote against any politician that argues that some goals are worth more deaths. It's not as though any of us are going to live for ever. 

\end{document}
