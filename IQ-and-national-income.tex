\documentclass{article}
\usepackage[margin=1in]{geometry}
\usepackage{booktabs}
\usepackage{hyperref}

\begin{document}

\title{Average IQ and GDP per capita by Country}
\author{}
\date{}
\maketitle

\section{Introduction}

The relationship between average IQ and GDP per capita is a topic of interest among researchers, as it may provide insights into the factors driving economic development.\footnote{For a comprehensive review of the literature, see \cite{LynnVanhanen2012} and \cite{Rindermann2018}.} The following table presents data on average IQ and GDP per capita for a selection of countries.

\begin{table}[h!]
\centering
\begin{tabular}{lrr}
\toprule
Country & Average IQ & GDP per capita (USD) \\
\midrule
Singapore & 108 & 64,133 \\
South Korea & 106 & 34,994 \\
Japan & 105 & 44,914 \\
Italy & 102 & 34,483 \\
Spain & 101 & 30,996 \\
United States & 98 & 69,862 \\
United Kingdom & 100 & 43,734 \\
France & 98 & 43,551 \\
Germany & 99 & 49,396 \\
Australia & 98 & 52,610 \\
Canada & 99 & 52,144 \\
China & 104 & 12,961 \\
India & 81 & 7,056 \\
Brazil & 87 & 14,845 \\
Russia & 96 & 23,549 \\
South Africa & 84 & 13,454 \\
Nigeria & 69 & 5,353 \\
Ethiopia & 63 & 2,423 \\
\bottomrule
\end{tabular}
\caption{Average IQ and GDP per capita by Country}
\label{tab:iq_gdp}
\end{table}

Sources:
\footnote{Average IQ: Various studies, including \cite{LynnVanhanen2012} and \cite{Rindermann2018}.}
\footnote{GDP per capita: World Bank (2022) \cite{WorldBank2022}.}

\section{Correlation between IQ and GDP per capita}

The data presented in Table \ref{tab:iq_gdp} suggests a strong positive correlation between average IQ and GDP per capita.\footnote{See \cite{JonesSchneider2010} for a discussion on the correlation between IQ and economic growth.} This correlation is consistent with the idea that cognitive abilities play a significant role in economic development. However, the direction of causation is not immediately clear.

\section{Direction of Causation}

There are two possible explanations for the observed correlation: (1) higher cognitive abilities lead to economic growth, or (2) economic growth leads to improved cognitive abilities through better education and nutrition.\footnote{For a discussion on the role of education and nutrition in cognitive development, see \cite{Nisbett2009}.} The first explanation suggests that countries with higher average IQs are more likely to have a skilled and innovative workforce, driving economic growth. The second explanation implies that as countries become wealthier, they are able to invest more in education and healthcare, leading to improved cognitive abilities.

\section{Natural Experiments}

To shed light on the direction of causation, researchers have looked to natural experiments where a single country or region has been partitioned, resulting in different economic outcomes.\footnote{Examples include North and South Korea, East and West Germany, and the Israeli-Palestinian divide \cite{HanushekWoessmann2015}.} These natural experiments suggest that economic conditions can have a significant impact on cognitive development. For instance, North Koreans have been found to have lower average IQs compared to South Koreans, likely due to differences in nutrition, healthcare, and education.\footnote{See \cite{Lynn2013} for a study on the intelligence of North Koreans.}

\section{Conclusion}

The correlation between average IQ and GDP per capita is a complex phenomenon that is influenced by a range of factors. While the direction of causation is not entirely clear, natural experiments suggest that economic conditions play a significant role in shaping cognitive development. Further research is needed to fully understand the relationship between cognitive abilities and economic growth.

\section{Bibliography}

\bibliographystyle{plain}
\bibliography{references}

\end{document}
